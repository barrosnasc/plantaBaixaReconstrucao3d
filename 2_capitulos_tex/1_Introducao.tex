                                          % INTRODUÇÃO %
%---------------------------------------------------------------------------------------------------------
\chapter{Introdução}\label{cp:introducao}
\ABNTEXchapterfont

% Contextualização problema em questão
% O que encontrei na bibliografia

% % Motivação do avanço tecnologico que motivou a utilização dessa técnica ao ínves das anteriores

% 2021Alibaba: Capítulo 2,¶1 e ¶2 :
% [...] Traditional methods [8, 9, 10] focus on directly processing low-level features. These systems produce a large number of hand-designed features and models. [...] Those systems mentioned above bring the problem of insufficient generalization ability. Thresholds and features are adjusted frequently by handcrafted operations instead of automatic methods. 
% With the development of deep learning techniques, the method of obtaining room structure has made significant process in generalization. Convolutional Neural Network (CNN) can create and extract advanced features to enhance the recognition performance of room elements [22, 29, 33]. Liu et al. [22] illustrate junctions of floor plans, for example, corners of walls, could be recognized by CNN [...]. However, the approach has limitations, for instance, it is not able to detect inclined walls. [...]. 

% - Justificativa digitalização dos documentos e utilização de
% dados quantitativos(monetário) na justificativo ( a lá redação enem )
% Dados quantitativos 
% MultiFloor: "Urban areas experience a steady growth, and over two-thirds of the population worldwide will be considered urbanized by the year 2050[1]" -> https://population.un.org/wup/assets/WUP2018-Report.pdf % Pegar a versão atualizada desse texto

% Dados qualitativos 
% MultiFloor: "may benefit multiple fields and subdomains, such as creating virtual twins of buildings [2], optimal evacuation path planning [7], and firefighting " 
% 2021Alibaba: "after the above rasterization process, designers cannot modify the structure of the room and redesign flexibly. Therefore, accurately recovering vectorized information from pixel images becomes an urgent problem to be solved."
% Precisa de uma referência que cita projetos de renovação

% Dados que justificam o uso da planta 3D ao invés da 2D
% ex: diminuição de custo, segurança pública

% Escrita
% Premissa: Há uma necessidade de utilizar gerar modelos 3D a partir de plantas 2D
A recuperação do modelo arquitetônico,\sigla{BIM}{Building Informational Model}, a partir de um planta baixa é uma necessidade que arquitetos e designers precisam para modificar e alterar o projeto \cite{lv2021residential}, atualmente os projetos \iffalse achismo sem referencia \fi são feitos com o auxilio do computador conhecido como \sigla{CAD}{Computer Aided Design}.
Há também a necessidade da criação do modelo 3D para planejar rotas de fuga e combate a incêndios \cite{kratochvila2024multi}.


\section{Objetivos}\label{cp:intro:obj}
Considerando a contextualização apresentada, nessa seção temos os objetivos gerais e específicos desse projeto de pesquisa.

\subsection{Objetivo Geral}\label{cp:intro:objgeral}
Desenvolver uma programa para gerar uma modelo 3D através de uma imagem de uma planta baixa de acordo com técnicas de processamento de imagem e detecção de característica que foram utilizadas nos artigos de referência. % das quais utilizando técnicas de baixo custo de hardware\cite{yang2022automated}.

\subsection{Objetivos Específicos}\label{cp:intro:objespec}
\begin{itemize}
  \item Revisão bibliográfica.
  % , para melhor entender as técnicas das referências, já que não está disponível o código fonte e com base na revisão recriar a arquitetura.
  \item Treinar o modelo de acordo com os datasets da literatura. % Datasets como o CubiCasa5K.
  \item Coletar das plantas baixas para serem processadas. % Digitalizar as plantas baixas disponíveis de Teresina.
  \item Fazer a inferência do modelo com as plantas baixas e a gerar o modelo 3D. % Executar o modelo e exportar para um formato de arquivo 3D: $.obj$ ou $.gltf$
  \item Exibir o modelo 3D gerado.%  Utilizando o framework ThreeJS \iffalse baseado em qual artigo que a escolhar do ThreeJS é valida? os artigos utilizam alguma forma de exibição mas não disseram qual e como especificamente \fi exibindo o desenho 3D, no navegador com o controle de órbita e $panning$.
\end{itemize}


\section{Organização do Trabalho}\label{cp:intro:organization}
Esta monografia está estruturada em 4 capítulos. O \autoref{cp:introducao} tem como objetivo mostrar ao leitor um panorama geral do trabalho, incluindo a contextualização ao qual a pesquisa está inserida, a justificativa e os objetivos geral e específicos. O \autoref{cp:refteory} apresenta o referencial teórico necessário para fundamentar os principais conceitos diretamente relacionados com a pesquisa a ser desenvolvida. No \autoref{cp:revisaoliteraria} são apresentados o protocolo de revisão literária, tal como os trabalhos relacionados com a proposta dessa pesquisa. No \autoref{cp:metodologia} são apresentados a classificação da pesquisa, o método a ser aplicado, o cronograma de desenvolvimento da monografia assim como os resultados que se espera alcançar através desta pesquisa. 

O trabalho é finalizado com a apresentação das referências bibliográficas que estruturaram a apresentação dos conceitos que constituem este estudo.





