                                          % INTRODUÇÃO %
%---------------------------------------------------------------------------------------------------------
\chapter{Introdução}\label{cp:introducao}
\ABNTEXchapterfont

% Contextualização problema em questão
% O que encontrei na bibliografia

% % Motivação do avanço tecnologico que motivou a utilização dessa técnica ao ínves das anteriores

% 2021Alibaba: Capítulo 2,¶1 e ¶2 :
% [...] Traditional methods [8, 9, 10] focus on directly processing low-level features. These systems produce a large number of hand-designed features and models. [...] Those systems mentioned above bring the problem of insufficient generalization ability. Thresholds and features are adjusted frequently by handcrafted operations instead of automatic methods. 
% With the development of deep learning techniques, the method of obtaining room structure has made significant process in generalization. Convolutional Neural Network (CNN) can create and extract advanced features to enhance the recognition performance of room elements [22, 29, 33]. Liu et al. [22] illustrate junctions of floor plans, for example, corners of walls, could be recognized by CNN [...]. However, the approach has limitations, for instance, it is not able to detect inclined walls. [...]. 

% - Justificativa digitalização dos documentos e utilização de
% dados quantitativos(monetário) na justificativo ( a lá redação enem )
% Dados quantitativos 
% MultiFloor: "Urban areas experience a steady growth, and over two-thirds of the population worldwide will be considered urbanized by the year 2050[1]" -> https://population.un.org/wup/assets/WUP2018-Report.pdf % Pegar a versão atualizada desse texto

% Dados qualitativos 
% MultiFloor: "may benefit multiple fields and subdomains, such as creating virtual twins of buildings [2], optimal evacuation path planning [7], and firefighting " 
% 2021Alibaba: "after the above rasterization process, designers cannot modify the structure of the room and redesign flexibly. Therefore, accurately recovering vectorized information from pixel images becomes an urgent problem to be solved."
% Precisa de uma referência que cita projetos de renovação

% Dados que justificam o uso da planta 3D ao invés da 2D
% ex: diminuição de custo, segurança pública

% Escrita
% Premissa: Há uma necessidade de utilizar gerar modelos 3D a partir de plantas 2D

O desenho arquitetônico contém vários tipos de objetos: paredes, portas, janelas, quartos, nomes de cômodos, móveis e suas dimensões, entre outros. 
Essas características fornecem ao leitor os meios para reproduzir, no mundo real, o que está representado no papel, seja em construções ou em simulações. 
A recuperação do modelo arquitetônico a partir de uma planta baixa é uma necessidade de arquitetos e designers que desejam modificar ou alterar um projeto \cite{lv2021residential}. Há também a necessidade de criação de modelos 3D para o planejamento de rotas de fuga e combate a incêndios \cite{kratochvila2024multi}.
%, atualmente os projetos \iffalse achismo sem referencia \fi são feitos com o auxilio do computador conhecido como \sigla{CAD}{Computer Aided Design}.

A transferência do desenho arquitetônico para a construção é estática e não pode ser alterada, o que gera um problema caso seja necessário realizar uma reforma. Torna-se, portanto, necessária a utilização de um ambiente que permita alterar a organização dos objetos na planta, algo viável no ambiente virtual \cite{lv2021residential}.
% transição de somente a extração de características do modelo 2D para um modelo CAD funcional para o modelo 3D, o que se ganha ao fazer esse tipo de transfomação?
Na ausência de um modelo virtual do desenho arquitetônico, é possível realizar sua reconstrução, seja manualmente, seja por meio do processamento de imagem. 
Neste projeto, será utilizada a técnica de processamento de imagem, que extrai objetos e características do documento digitalizado, transfere-os para um ambiente virtual, processa-os para gerar uma estrutura de dados manipulável e, por fim, transforma-os em um modelo 3D \cite{yang2022automated}.

As primeiras técnicas de processamento de imagem para extração de características de plantas baixas utilizavam algoritmos heurísticos para obter informações das imagens. Com o avanço da ciência, observou-se que técnicas de aprendizado profundo apresentam desempenho superior \cite{lv2021residential}. Essas técnicas exigem \textit{datasets} para o treinamento dos modelos de predição, como o CubiCasa5K \cite{cubiCasa5K}, que contém 5 mil imagens de plantas baixas de residências da Finlândia.

\section{Objetivos}\label{cp:intro:obj}
Considerando a contextualização apresentada, nesta seção são apresentados os objetivos gerais e específicos deste projeto de pesquisa.

\subsection{Objetivo Geral}\label{cp:intro:objgeral}
Desenvolver um programa que utilize técnicas de processamento de imagem e detecção de símbolos aplicadas a plantas baixas, com base nas abordagens apresentadas nos artigos de referência, para gerar uma estrutura de dados que possibilite a criação de um modelo 3D. % das quais utilizando técnicas de baixo custo de hardware\cite{yang2022automated}.

\subsection{Objetivos Específicos}\label{cp:intro:objespec}
\begin{itemize}
  \item Revisar a bibliografia para compreender melhor as técnicas apresentadas nos artigos de referência e, com base nessa revisão, recriar a arquitetura utilizada.
  \item Escolher a arquitetura a ser utilizada, priorizando aquela com melhor compatibilidade com os requisitos de hardware disponíveis.
  \item Treinar o modelo de acordo com os \textit{datasets} da literatura.
  \item Coletar plantas baixas, preferencialmente de Teresina, para serem processadas e utilizadas na validação do modelo. % Digitalizar as plantas baixas disponíveis de Teresina.
  \item Fazer a inferência do modelo com as plantas baixas selecionadas. % Executar o modelo e exportar para um formato de arquivo 3D: $.obj$ ou $.gltf$
  \item Reconstruir e exibir o modelo 3D, para visualizar e comparar com a planta original.%  Utilizando o framework ThreeJS \iffalse baseado em qual artigo que a escolhar do ThreeJS é valida? os artigos utilizam alguma forma de exibição mas não disseram qual e como especificamente \fi exibindo o desenho 3D, no navegador com o controle de órbita e $panning$.
\end{itemize}


\section{Organização do Trabalho}\label{cp:intro:organization}
Esta monografia está estruturada em 4 capítulos. O \autoref{cp:introducao} tem como objetivo mostrar ao leitor um panorama geral do trabalho, incluindo a contextualização ao qual a pesquisa está inserida, a justificativa e os objetivos geral e específicos. O \autoref{cp:refteory} apresenta o referencial teórico necessário para fundamentar os principais conceitos diretamente relacionados com a pesquisa a ser desenvolvida. No \autoref{cp:revisaoliteraria} são apresentados o protocolo de revisão literária, tal como os trabalhos relacionados com a proposta dessa pesquisa. No \autoref{cp:metodologia} são apresentados a classificação da pesquisa, o método a ser aplicado, o cronograma de desenvolvimento da monografia, o fluxograma das etapas de desenvolvimento assim como os resultados que se espera alcançar através desta pesquisa. 

O trabalho é finalizado com a apresentação das referências bibliográficas que estruturaram a apresentação dos conceitos que constituem este estudo.

