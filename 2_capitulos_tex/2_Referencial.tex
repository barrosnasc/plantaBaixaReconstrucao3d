\chapter{Referencial Teórico}\label{cp:refteory}
\ABNTEXchapterfont
% Justificativa das tecnologias, comparação 
%Este capítulo é um componente importante para esse estudo, pois fornece uma base  explorando as principais teorias, conceitos e definições existentes sobre as áreas de conhecimento que motivaram a elaboração deste projeto de pesquisa e que guiarão a execução do percurso metodológico.

%Escreva sobre o referencial teórico do seu trabalho

% Qual é o tipo de problema
% Utilizando o artigo do 3DPlanNet

Como já citado o  desenho arquitetônico contém vários tipos de objetos: paredes, portas, janelas, quartos, nome do cômodo, móveis, as suas dimensões,  entre outros% , e essas informações serão compreendidas pelo computador. 
mas cada autor escolheu por ênfase em algum desses aspectos.

O processamento da imagem gera alguns problemas sendo que antes foram solucionados por algoritmos heurísticos, mas atualmente as soluções usam técnicas de aprendizagem profunda \iffalse referênciar? \fi e essa mudança mostrou uma melhora na acurácia na detecção dos objetos \cite{3dplanet2021}.

Um dos problemas que isso gera é a necessidade da criação de datasets devidamente anotados para treinar os modelos de aprendizagem profundas.

\subsection{Datasets}
Aqui está uma lista com cada autor e o seus datasets que foram utilizados:
\begin{itemize}
    \item \cite{3dplanet2021} utiliza dataset próprio de 30 imagens.
    \item \cite{lv2021residential} utiliza o próprio dataset "Residential Floor Plan" composto de 7000 imagens.
    \item \cite{yang2022automated}  utiliza o dataset R2V\cite{R2V_ICCV_2017} e  R3D\cite{R3D_2015_CVPR} e plantas de casa de Beike.% Site de casas www.ke.com Página 9 "EXPERIMENTS AND ANALYSIS" 
    \item \cite{kratochvila2024multi} e \cite{barreiro2023automatic} utilizaram o dataset \cite{cubiCasa5K} nos seus experimentos.
\end{itemize}

% Utilizar o resumo/conclusão de cada artigo, e discutir sobre os métodos e os achados e comentar sobre os artigos mostrando que ainda há algo a ser feito 

% Como as pessoas estão resolvendo os problemas
% sugestão de tabelinha de como comparar as diferentes soluções ajuda consideravelmente no entendimento do trabalho

\section{Considerações Finais}
Se necessário escreva consideracoes finais e faça o link com o que será visto no proximo capítulo
