\chapter{Referencial Teórico}\label{cp:refteory}
\ABNTEXchapterfont
% Justificativa das tecnologias, comparação 
%Este capítulo é um componente importante para esse estudo, pois fornece uma base  explorando as principais teorias, conceitos e definições existentes sobre as áreas de conhecimento que motivaram a elaboração deste projeto de pesquisa e que guiarão a execução do percurso metodológico.

%Escreva sobre o referencial teórico do seu trabalho

% Qual é o tipo de problema
% Utilizando o artigo do 3DPlanNet

Como já citado, o desenho arquitetônico contém vários tipos de objetos: paredes, portas, janelas, quartos, nomes de cômodos, móveis e suas dimensões, entre outros. Cada autor opta por dar ênfase a determinados desses aspectos. Seguindo a abordagem de \citeonline{kratochvila2024multi} e \citeonline{lv2021residential}, a implementação é dividida em duas etapas: reconhecimento e reconstrução.
Na etapa de reconhecimento ocorre o processamento de imagem, gerando um modelo intermediário, e na reconstrução são realizados o pós-processamento e a criação do modelo 3D, com pelo menos os objetos de parede, porta e janela.

No reconhecimento há a detecção de paredes, portas e janelas, sendo criada uma rede de vértices e arestas das paredes \cite{3dplanet2021}. Dependendo do autor, pode haver detecção dos outros tipos de objetos presentes na planta.
Na reconstrução é feito o pós-processamento para normalizar e, em seguida, é criado o modelo 3D usando tamanhos padrão para os objetos, como altura e largura da parede. Caso a técnica permita detectar outros elementos na etapa anterior, eles são adicionados ao modelo 3D.

\section{Desenho Arquitetônico e Planta baixa}
De acordo com a ABNT NBR 6492:2021{\cite{nbr6492}, o desenho arquitetônico é composto por diversos elementos gráficos e simbólicos que representam a edificação, como paredes, portas, janelas, escadas, mobiliários e cotas (indicadores de tamanho). Esses elementos seguem convenções técnicas que padronizam escalas, espessuras de linha e simbologias, facilitando a interpretação e a comunicação entre profissionais da área. A norma utiliza o termo “planta”, enquanto o termo “planta baixa” é usado na prática.
\section{Processamento de Imagem}
O processamento de imagens “considera a manipulação de imagens depois de capturadas por dispositivos imageadores” \cite[p.~4]{conci_computacao_2021_v2}, podendo envolver redução de ruído, realce e transformação de pixels. Essas técnicas são fundamentais como etapa inicial para extração de padrões e reconhecimento de estruturas em projetos arquitetônicos digitalizados.
\subsection{Reconhecimento de Padrões}
O reconhecimento de padrões exige que o algoritmo aprenda a distinguir classes visuais distintas a partir de exemplos previamente identificados. O livro explica que, na classificação supervisionada, emprega-se “um conjunto-padrão de objetos conhecidos pertencentes a diferentes classes” \cite[p.~206]{conci_computacao_2021_v2} e esse conjunto de dados justifica o uso dos \textit{datasets}.
\subsection{Segmentação de Imagem}
A segmentação de imagens consiste em dividir a imagem em regiões que possuam propriedades semelhantes, permitindo isolar partes relevantes para análise posterior. De acordo com o livro “a ideia geral em muitos dos métodos de segmentação é agrupar, de alguma forma, pixels ou grupos de pixels com mesma propriedade” \cite[p.~201]{conci_computacao_2021_v2}.
 \section{\textit{Datasets}}
Conjuntos de dados anotados são utilizados para o treinamento de algoritmos de detecção e segmentação, além de servirem como base comparativa entre diferentes técnicas. Essa comparação é empregada para verificar quantitativamente a eficiência e qualidade desses algoritmos.
% \section{Trabalhos Relacionados}
% \cite{Goncalves2019}, utilizou de algoritmos heurísticos para gerar um modelo 3D a partir de desenhos manuais, levou o  

% colocar a figura do artigo

% O processamento da imagem gera alguns problemas sendo que antes foram solucionados por algoritmos heurísticos, mas atualmente as soluções usam técnicas de aprendizagem profunda \iffalse referênciar? \fi e essa mudança mostrou uma melhora na acurácia na detecção dos objetos \cite{3dplanet2021}.

% Um dos problemas que isso gera é a necessidade da criação de datasets devidamente anotados para treinar os modelos de aprendizagem profundas.

% Utilizar o resumo/conclusão de cada artigo, e discutir sobre os métodos e os achados e comentar sobre os artigos mostrando que ainda há algo a ser feito 

% Como as pessoas estão resolvendo os problemas
% sugestão de tabelinha de como comparar as diferentes soluções ajuda consideravelmente no entendimento do trabalho

\section{Considerações Finais}
Este capítulo apresentou os principais conceitos que fundamentam o desenvolvimento da proposta: os elementos do desenho arquitetônico, o uso de técnicas de processamento de imagem, reconhecimento de padrões, segmentação, e a importância dos datasets anotados.
