\chapter{Referencial Teórico}\label{cp:refteory}
\ABNTEXchapterfont
% Justificativa das tecnologias, comparação 
%Este capítulo é um componente importante para esse estudo, pois fornece uma base  explorando as principais teorias, conceitos e definições existentes sobre as áreas de conhecimento que motivaram a elaboração deste projeto de pesquisa e que guiarão a execução do percurso metodológico.

%Escreva sobre o referencial teórico do seu trabalho

% Qual é o tipo de problema
% Utilizando o artigo do 3DPlanNet

Como já citado o desenho arquitetônico contém vários tipos de objetos: paredes, portas, janelas, quartos, nome do cômodo, móveis, as suas dimensões, entre outros, cada autor escolheu por ênfase em algum desses aspectos. Utilizando a divisão de \citeonline{kratochvila2024multi} e \citeonline{lv2021residential}, cada técnica é dividido em duas etapas: reconhecimento e reconstrução, reconhecimento é onde ocorre o processamento de imagem gerando um modelo intermediário, e na reconstrução é o pós-processamento e a criação do modelo 3D, com os elementos bases de paredes, portas e janelas.

No reconhecimento há a detecção de paredes, portas e janelas, onde é criada uma rede de vértices e arestas das paredes, também pode haver detecção dos outros tipos de objetos, depente do autor.
Na reconstrução é feita o pós-processamento para diminuir o erro, é então criado o modelo 3d usando tamanhos padrões para os objetos, como altura e largura da parede, caso a técnica tenha detectados objetos na etapa anterior serão adicionados no modelo 3d. Nenhum dos trabalhos citados tem o código fonte disponível.

\section{Trabalhos Relacionados}

\citeonline{kratochvila2024multi} desenvolveu dois algoritmos de segmentação CAB1 e CAB2, que utiliza \textit{Attention Mechanism} composto por \textit{Channel attention module} e \textit{Spatial attention module}. Na etapa de reconstrução, vetorizou cada categoria de pixel segmentados e gerou o modelo 3D. Utilizou o 560 imagens escolhidas do dataset CubiCasa5K. \autoref{fig:kratochvila2024} mostra a visão geral do próprio autor.

\citeonline{lv2021residential}. Faz da detecção de área de interesse, texto e de simbolos (móveis, pias) YOLOv4 \cite{bochkovskiy2020yolov4}.
Calcula o tamanho da escala do desenho, pixel por mm , enquanto outros como o \cite{3dplanet2021} utilizada de um tamanho padrão.Utilizou dataset próprio %\sigla{RPF}{Residential Floor Plan}
 com 7000 imagens de residencias chinesas. \autoref{fig:lv2021} mostra a visão geral do autor.

\citeonline{barreiro2023automatic}.De acordo com o autor, focar somente em paredes, portas e janelas, pois as anotações do tipo de cômodo é muito específico de cada dataset e não pode ser generalizada. A sequência de ações é: detecção de simbolos para detectar portas e janelas, segmentação de paredes utilizando arquitetura FPN com o backbone ResNet similar ao \cite{lv2021residential}.

\citeonline{3dplanet2021}, citado por \citeonline{kratochvila2024multi}, utilizou o algoritmo de \cite{ijgi9020065} para detectar o centro das paredes. Utilizou a api do Tensorflow para a detecção de objetos: tipo de cômodo, portas e janelas. Utilizou 30 imagens escolhidas aleatoriamente providas do dataset de \citeonline{ijgi9020065} que contém 110,000 imagens.


\begin{figure}[H]
\centering
\caption{Arquiteturas de Kratochvila et al. (2024) e Lv et al. (2021)}
\begin{subfigure}{0.47\textwidth}
  \centering
  \includegraphics[width=\linewidth]{imagens/kratochvila2024multi_figure4.png}
  \caption{Kratochvila et al. (2024)}
  \label{fig:kratochvila2024}
\end{subfigure}
\hfill
\begin{subfigure}{0.47\textwidth}
  \centering
  \includegraphics[width=\linewidth]{imagens/Lv_2021_figure2.png}
  \caption{Lv et al. (2021)}
  \label{fig:lv2021}
\end{subfigure}
\legend{Fonte: (a) \cite{kratochvila2024multi}, p. 5; (b) \cite{lv2021residential}, p. 16719.}
\label{fig:arquiteturas1}
\end{figure}

\begin{figure}[H]
\centering
\caption{Arquiteturas de Barreiro et al., (2023), e Park; Kim, (2021)}
\begin{subfigure}{0.47\textwidth}
  \centering
  \includegraphics[width=\linewidth]{imagens/barreiro2023automatic_figure1.png}
  \caption{Da direita para esquerda, planta baixa, mascara segmentada, resultados vetorizados, modelo 3d }
  \label{fig:barreiro2023automatic}
\end{subfigure}
\hfill
\begin{subfigure}{0.47\textwidth}
  \centering
  \includegraphics[width=\linewidth]{imagens/3dplanet2021_figure1.png}
  \caption{Lv et al. (2021)}
  \label{fig:3dplanet2021}
\end{subfigure}
\legend{Fonte: (a) \cite{barreiro2023automatic}, p. 3; (b) \cite{3dplanet2021}, p. 3.}
\label{fig:arquiteturas2}
\end{figure}


\end

% colocar a figura do artigo

% O processamento da imagem gera alguns problemas sendo que antes foram solucionados por algoritmos heurísticos, mas atualmente as soluções usam técnicas de aprendizagem profunda \iffalse referênciar? \fi e essa mudança mostrou uma melhora na acurácia na detecção dos objetos \cite{3dplanet2021}.

% Um dos problemas que isso gera é a necessidade da criação de datasets devidamente anotados para treinar os modelos de aprendizagem profundas.

% Utilizar o resumo/conclusão de cada artigo, e discutir sobre os métodos e os achados e comentar sobre os artigos mostrando que ainda há algo a ser feito 

% Como as pessoas estão resolvendo os problemas
% sugestão de tabelinha de como comparar as diferentes soluções ajuda consideravelmente no entendimento do trabalho

\section{Considerações Finais}
Os trabalhos analisados utilizam de técnicas de processamento de imagem e de reconstrução 3d, mas falta a disponibilização do código fonte e dos datasets utilizados, o que dificulta a sua reprodução.
