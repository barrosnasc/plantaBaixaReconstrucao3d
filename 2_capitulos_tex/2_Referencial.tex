\chapter{Referencial Teórico}\label{cp:refteory}
\ABNTEXchapterfont
% Justificativa das tecnologias, comparação 
%Este capítulo é um componente importante para esse estudo, pois fornece uma base  explorando as principais teorias, conceitos e definições existentes sobre as áreas de conhecimento que motivaram a elaboração deste projeto de pesquisa e que guiarão a execução do percurso metodológico.

%Escreva sobre o referencial teórico do seu trabalho

% Qual é o tipo de problema
% Utilizando o artigo do 3DPlanNet
O desenho arquitetônico descreve vários tipos de objetos: parede, portas, janela, quartos, dimensões escritas no documento, nome do cômodo, entre outros, que é então escaneado para gerar um imagem digital que o algoritmo precisa processar e transformar em uma estrutura de dados que seja capaz de manipular.

Antes os problemas que conversão gera foram solucionados por algoritmos heurísticos, mas que atualmente as soluções usam técnicas de aprendizagem profunda \iffalse referênciar? \fi e essa mudança mostrou uma melhora na acurácia na detecção dos objetos \cite{3dplanet2021}.

Um dos problemas que isso gera é a necessidade da criação de datasets devidamente anotados para treinar os modelos de aprendizagem profundas. A técnica de \cite{3dplanet2021} se mostra interessante por conseguir com 30 imagens anotadas gerar um modelo que \cite{kratochvila2024multi} diz é uma das técnicas de estado da arte.

% Ajudar com base na literatura que ainda existe na 

% Utilizar o resumo/conclusão de cada artigo, e discutir sobre os métodos e os achados e comentar sobre os artigos mostrando que ainda há algo a ser feito 

% Como as pessoas estão resolvendo os problemas
% sugestão de tabelinha de como comparar as diferentes soluções ajuda consideravelmente no entendimento do trabalho



% Motivo de não usar LiDAR: \citeonline{yang2022automated}

% \section{Figuras no texto}\label{cp:refteory:figuras}

% Meu texto a \autoref{figs/batepapo} mostra... figura \ref{figs/batepapo}

% \begin{figure}[H]
%     \centering
%     \caption{Comunicação Interativa entre Usuários.}
%     \includegraphics[width=16cm, height=12cm]{imagens/snowballing.png}
%     \legend{Fonte: \citeonline{snowballing4guidelines}.}
%     \label{figs/batepapo}  % Adicionando o label para a figura
% \end{figure}

\section{Considerações Finais}
Se necessário escreva consideracoes finais e faça o link com o que será visto no proximo capítulo
