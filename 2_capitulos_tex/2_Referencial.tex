\chapter{Referencial Teórico}\label{cp:refteory}
\ABNTEXchapterfont
% Justificativa das tecnologias, comparação 
%Este capítulo é um componente importante para esse estudo, pois fornece uma base  explorando as principais teorias, conceitos e definições existentes sobre as áreas de conhecimento que motivaram a elaboração deste projeto de pesquisa e que guiarão a execução do percurso metodológico.

%Escreva sobre o referencial teórico do seu trabalho

% Qual é o tipo de problema
% Utilizando o artigo do 3DPlanNet

Como já citado o desenho arquitetônico contém vários tipos de objetos: paredes, portas, janelas, quartos, nome do cômodo, móveis, as suas dimensões,  entre outros% , e essas informações serão compreendidas pelo computador. 
mas cada autor escolheu por ênfase em algum desses aspectos.

\section{Trabalhos Encontrados}

%Diferentes autores escolheram focar em características diferentes do desenho arquitetônico.

%Diferentes autores utilizaram diferentes técnicas.

%Diferentes autores utilizaram hardwares diferentes para fazer o treinamento.

%Utilizando a divisão de \citeonline{kratochvila2024multi}, cada técnica cada autor é dividido em duas etapas: reconhecimento e reconstrução. Reconhecimento é onde ocorre o processamento de imagem gerando um modelo intermediário, e na reconstrução é a criação do modelo 3D.

\citeonline{kratochvila2024multi} desenvolveu dois algoritmos de segmentação CAB1 e CAB2, que utiliza \textit{Attention Mechanism} que é composto por \textit{Channel attention module} e \textit{Spatial attention module}. Na etapa de reconstrução vetorizou cada categoria de pixel segmentados e gerou o modelo 3D. Utilizou o 560 imagens escolhidas do dataset CubiCasa5K. \autoref{fig:kratochvila2024} mostra a visão geral do próprio autor.



% colocar a figura do artigo

\citeonline{3dplanet2021} utilizou o algoritmo de \cite{ijgi9020065} para detectar o centro das paredes. Utilizou a api do Tensorflow para a detecção de objetos: tipo de cômodo, portas e janelas. Utilizou 30 imagens escolhidas aleatoriamente providas do dataset de \citeonline{ijgi9020065} que contém 110,000 imagens.

% colocar a figura do artigo

\citeonline{lv2021residential}. Faz da detecção de área de interesse, texto e de simbolos (móveis, pias) YOLOv4 \cite{bochkovskiy2020yolov4}.
Calcula o tamanho da escala do desenho, enquanto outros como o \cite{3dplanet2021} utilizada de um tamanho padrão.Utilizou dataset próprio \sigla{RPF}{Residential Floor Plan} com 7000 imagens de residencias chinesas. \autoref{fig:lv2021} mostra a visão geral do autor.

\begin{figure}[H]
\centering
\caption{Arquiteturas de Kratochvila et al. (2024) e Lv et al. (2021)}
\begin{subfigure}{0.47\textwidth}
  \centering
  \includegraphics[width=\linewidth]{imagens/kratochvila2024multi_figure4.png}
  \caption{Kratochvila et al. (2024)}
  \label{fig:kratochvila2024}
\end{subfigure}
\hfill
\begin{subfigure}{0.47\textwidth}
  \centering
  \includegraphics[width=\linewidth]{imagens/Lv_2021_figure2.png}
  \caption{Lv et al. (2021)}
  \label{fig:lv2021}
\end{subfigure}
\legend{Fonte: (a) \cite{kratochvila2024multi}, p. 5; (b) \cite{lv2021residential}, p. 16719.}
\label{fig:comparacao_arquiteturas}
\end{figure}

\end

% colocar a figura do artigo

% O processamento da imagem gera alguns problemas sendo que antes foram solucionados por algoritmos heurísticos, mas atualmente as soluções usam técnicas de aprendizagem profunda \iffalse referênciar? \fi e essa mudança mostrou uma melhora na acurácia na detecção dos objetos \cite{3dplanet2021}.


% Um dos problemas que isso gera é a necessidade da criação de datasets devidamente anotados para treinar os modelos de aprendizagem profundas.

% \subsection{Datasets}
% Aqui está uma lista com cada autor e o seus datasets que foram utilizados:
% \begin{itemize}
%     \item \cite{kratochvila2024multi} e \cite{barreiro2023automatic} utilizaram o dataset \cite{cubiCasa5K} nos seus experimentos, que contém 5000 mil imagens de plantas residenciais da Finlândia.
%     \item \cite{yang2022automated}  utiliza o dataset R2V\cite{R2V_ICCV_2017} e  R3D\cite{R3D_2015_CVPR} e plantas de casa de Beike.% Site de casas www.ke.com Página 9 "EXPERIMENTS AND ANALYSIS" 
%     \item \cite{lv2021residential} utiliza o próprio dataset "Residential Floor Plan" composto de 7000 imagens.
%     \item \cite{3dplanet2021} utiliza dataset próprio de 30 imagens.
% \end{itemize}
% O CubiCasa5K foi desenvolvido na Finlândia, então há anotações como banco de sauna, lareira, banheira, chaminé. %  'Sauna bench', 'Fire Place', 'Bathtub', 'Chimney'
% O Residential Floor Plan \cite{lv2021residential} foi desenvolvido na China, dando enfase na detecção de escala, região de interesse. 
% O \cite{R2V_ICCV_2017}

% Utilizar o resumo/conclusão de cada artigo, e discutir sobre os métodos e os achados e comentar sobre os artigos mostrando que ainda há algo a ser feito 

% Como as pessoas estão resolvendo os problemas
% sugestão de tabelinha de como comparar as diferentes soluções ajuda consideravelmente no entendimento do trabalho

\section{Considerações Finais}
Se necessário escreva consideracoes finais e faça o link com o que será visto no proximo capítulo
