\chapter{Proposta e Metodologia}\label{cp:metodologia}
% Para a formulação dos procedimentos de pesquisa, retoma-se o objetivo proposto neste estudo:

Este capítulo visa relatar o processo metodológico utilizado para desenvolvimento do estudo (\autoref{cp:intro:metodos}), o cronograma de tarefas (\autoref{cp:proposta:cronograma}) e os resultados que se espera obter ao fim do prazo de execução da monografia (\autoref{resultados}).

\section{Método e Procedimentos de Pesquisa}\label{cp:intro:metodos}

O projeto é treinar um modelo com base na literatura que faça a detecção de características das plantas baixas da cidade de Teresina com isso então criar um modelo 3D e exibir em um site comparando o resultado com a planta baixa original. 

Com base nessas informações, do ponto de vista dos objetivos, esse estudo pode ser considerado uma pesquisa qualitativa. As etapas de desenvolvimento do projeto de pesquisa  são:

\begin{enumerate}   
  \item \textbf{Definição da arquitetura:} Selecionar a arquitetura com melhor compatibilidade os requisitos de hardware disponíveis e o resultado esperado.
  \item \textbf{Treinamento do Modelo:} Realizar o treinamento do modelo com base nos datasets e parâmetros encontrados na literatura literatura.
  \item \textbf{Coleta das plantas baixas:} Coletar as plantas baixas de Teresina digitalizadas. % Digitalizar as plantas baixas disponíveis de Teresina.
  \item \textbf{Inferência do modelo:} Com o modelo treinado e com as plantas baixas coletas. % Executar o modelo e exportar para um formato de arquivo 3D: $.obj$ ou $.gltf$   
  \item \textbf{Reconstrução e exibição do modelo 3D:} Exibir o modelo 3D em um site utilizando uma biblioteca de renderização 3D\footnote{\textit{Three.js}: Biblioteca 3D em \textit{JavaScript} \url{https://threejs.org/}}, exibindo lado a lado a planta baixa original e a planta baixa processada.%  Utilizando o framework ThreeJS \iffalse baseado em qual artigo que a escolhar do ThreeJS é valida? os artigos utilizam alguma forma de exibição mas não disseram qual e como especificamente \fi exibindo o desenho 3D, no navegador com o controle de órbita e $panning$.
\end{enumerate}

\section{Cronograma de Pesquisa}\label{cp:proposta:cronograma}
O \autoref{cronograma} apresenta o cronograma mensal de atividades necessárias para desenvolvimento da pesquisa proposta: 

% Criação de figura 

\begin{quadro}[!htb]
    \centering
    \caption{Cronograma de desenvolvimento das atividades de pesquisa.}\label{cronograma}
    \begin{tabular}{|l|c|c|c|c|}
    \hline
    & \multicolumn{4}{|c|}{2026} \\ \cline{2-5}
    \textbf{Atividades}&  Janeiro&  Fevereiro&  Março& Abril\\ \hline  
    1. Revisão Bibliografica&  X&  X&  &  \\ \hline  
    2. Definição da arquitetura&  X&  &  & \\ \hline  
    3. Coleta das plantas baixas&  X&  X&  & \\ \hline  
    4. Treinamento do modelo &  X&  X&  & \\ \hline  
    5. Reconstrução e exibição do modelo 3D&  X&  X&  X& \\ \hline  
    6. Escrita&  &  X&  X& X\\ \hline  
    7. Defesa&  &  &  & X\\ \hline      
    \end{tabular}

    \vspace{1cm}    
\end{quadro}

\section{Resultados Esperados}\label{resultados}

Espera-se, ao final do desenvolvimento deste projeto, obter um modelo treinado capaz de realizar a detecção automática das características presentes em plantas baixas, aplicando técnicas de processamento de imagem e aprendizado profundo descritas na literatura.  

O sistema deverá ser capaz de processar plantas digitalizadas da cidade de Teresina e, a partir dos resultados da inferência, gerar um modelo tridimensional correspondente. Essa reconstrução 3D será exibida em um ambiente web desenvolvido, com uma visualização comparativa entre a planta original e a planta processada.  
