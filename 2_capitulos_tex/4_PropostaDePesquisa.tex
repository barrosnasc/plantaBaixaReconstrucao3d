\chapter{Proposta e Metodologia}\label{cp:metodologia}
Para a formulação dos procedimentos de pesquisa, retoma-se o objetivo proposto neste estudo:
Irei usar a plaforma do Google Collab para o treinamento e inferência
Irei usar o dataset CubiCasa5k
Irei usar o Pytorch para a criação do modelo
Irei usar o 
\begin{quotation}
\textit{ --  }
\end{quotation}

Este capítulo visa relatar o processo metodológico utilizado para desenvolvimento do estudo (\autoref{cp:intro:metodos}), o cronograma de tarefas (\autoref{cp:proposta:cronograma}) e os resultados que se espera obter ao fim do prazo de execução da monografia (\autoref{resultados}).

\section{Método e Procedimentos de Pesquisa}\label{cp:intro:metodos}
 ....

Com base nessas informações, do ponto de vista dos objetivos, esse estudo pode ser considerado uma pesquisa .... As etapas de desenvolvimento do projeto de pesquisa ...:

\begin{enumerate} 
\item ...

\end{enumerate}

\section{Cronograma de Pesquisa}\label{cp:proposta:cronograma}
O \autoref{cronograma} apresenta o cronograma mensal de atividades necessárias para desenvolvimento da pesquisa proposta ...

Criação de figura 

\begin{quadro}[!htb]
    \centering
    \caption{Cronograma de desenvolvimento das atividades de pesquisa.}\label{cronograma}
    \begin{tabular}{|l|c|c|c|c|}
    \hline
    & \multicolumn{4}{|c|}{2025} \\ \cline{2-5}
    \textbf{Atividades}&  Janeiro&  Fevereiro&  Março& Abril\\ \hline  
    1. Rev. Biblio.&  X&  X&  X& X \\ \hline  
    2. Map. teo. BD&  X&  X&  & \\ \hline  
    3. Exp. Elementos Design&  X&  X&  & \\ \hline  
    4. Def. Elementos &  X&  X&  & \\ \hline  
    5. Prop. Métricas&  X&  X&  & \\ \hline  
    6. Desen. do Framework&  &  X&  X& \\ \hline  
    7. Util. Framework&  X&  &  X& X\\ \hline  
    8. Escrita&  &  X&  X& X\\ \hline  
    9. Defesa&  &  &  & X\\ \hline 
    \end{tabular}

    \vspace{1cm}
    
\end{quadro}

\section{Resultados Esperados}\label{resultados}
 

