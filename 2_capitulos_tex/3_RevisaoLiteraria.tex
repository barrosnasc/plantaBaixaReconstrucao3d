\chapter{Revisão Literária}\label{cp:revisaoliteraria}
Foi realizada uma busca exploratória utilizando a ferramenta Google Scholar para encontrar um artigo que fosse relevante ao tema e foi encontrado \cite{kratochvila2024multi}, com o conteúdo desse foi gerado e refinado a string de busca.

\section{Busca Bibliográfica}\label{cp:refteory:figuras}
A busca nas bibliotecas: arXiv e IEEE Xplore foi feita em setembro de 2025, abrangendo o período de 2021 a 2025, com as strings de busca mostrado a seguir:
   
\subsection{arXiv}
A biblioteca Python \texttt{arxiv}~\cite{pypi_arxiv}, foi utilizada para realizar a busca no 
repositório arXiv.


String de Busca arXiv: 
\begin{verbatim}
('(all:"floor plan" OR all:floorplan OR all:floorplans 
    OR all:"architectural layout" OR all:"building layout") ' 
'AND (all:"3D reconstruction" OR all:"3D model" 
    OR all:"layout reconstruction" OR all:"vectorization" 
    OR all:"raster-to-vector" OR all:"plan-to-3D" 
    OR all:"scene generation" OR all:"3D scenes" OR all:Plan2Scene)'
'AND (all:"deep learning" OR all:"semantic segmentation" 
OR all:"multi-task" OR all:"multi-task learning" 
    OR all:"graph neural network" OR all:GNN 
    OR all:CNN OR all:YOLO OR all:DeepLab 
    OR all:"geometric optimization") '
'AND submittedDate:[20190101 TO 20251231] '
'ANDNOT (all:"point cloud" OR all:LiDAR OR all:LIDAR  
    OR all:SLAM OR all:mapping OR all:scan 
    OR all:BIM OR all:robot OR all:navigation 
    OR all:"scene understanding" OR all:"depth estimation")')
\end{verbatim}
\subsection{IEEE Xplore}
String de Busca IEEE Xplore:

O site do IEEE Xplore foi usado para fazer a a busca com a string de busca.
\begin{verbatim}
("3D reconstruction" OR "3D modeling" OR "3D generation") 
AND ("floor plan" OR "architectural plan") 
AND ("deep learning" OR "semantic segmentation")
\end{verbatim}

%Artigos do IEEE Xplore que não tive acesso foram excluidos
\section{Síntese dos Resultados da Busca}

O quadro a seguir apresenta o quantitativo de resultados obtidos e filtrados durante as etapas de busca e triagem.

\begin{quadro}[H]
\centering
\caption{Quantitativo das etapas de busca 
e triagem bibliográfica (2021\\-\\2025)}
\label{quadro:busca_invertido}
\begin{tabular}{|l|c|c|c|c|}
\hline
\textbf{Etapa} & \textbf{arXiv} & \textbf{IEEE Xplore}  & \textbf{Total Consolidado} \\
\hline
Total Recuperado & 56 & 14 & 70 \\
Após Título e/ou Abstract & 11 & 3 & 14 \\
Após Leitura Completa & 2 & 1 & 3 \\
\hline
Incluídos & 2 & 1 & 3 \\
\hline
\end{tabular}
\legend{Fonte: Autor}
\end{quadro}

Os artigos selecionados foram: \citeonline{kratochvila2024multi} e \citeonline{barreiro2023automatic} do ArXiv e \citeonline{lv2021residential} do IEEE Xplore. 

Os outros artigos analisados foram encontrados a partir das referências dos artigos principais da busca.

% precisa apontar para o parágrafo da fonte? E como que escreve isso LaTeX

% Este capítulo define o protocolo de revisão utilizado para obtenção de trabalhos com propostas semelhantes as definidas nesse estudo, tal como a fundamentação teórica necessária para os conceitos apresentados no \autoref{cp:refteory}.

% \section{Protocolo de Revisão}\label{cp:revisao:protocolo}

% Foi usando a ferramenta Rayyan\cite{Rayyn} para fazer a filtragem dos artigos.

% \subsection{Critérios de Exclusão}
% \begin{itemize}
%     \item LiDAR - \textit{light detection and ranging}.
%     \item Slam  - \textit{Simultaneous localization and mapping}.
%     \item Point Clouds - Nuvem de pontos.    
% \end{itemize}

\section{Considerações Finais}
% Protocolo de obtenção dos artigos
% Frameworks com abordagens diferentes com foco na promoção do engajamento e motivação do usuário.
% Neste capítulo foram apresentados o protocolo de revisão e os trabalhos relacionados com os \nameref{cp:intro:obj} deste estudo. ....

Na busca bibliográfica \cite{kratochvila2024multi} se tornou-se o artigo principal de referência, pois cita e relaciona os demais artigos encontrados. A partir dele, foi possível identificar os \textit{datasets} e as técnicas predominantes aplicadas atualmente na reconstrução 3D a partir de plantas baixas..