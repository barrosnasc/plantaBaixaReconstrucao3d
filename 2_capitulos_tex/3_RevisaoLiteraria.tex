\chapter{Revisão Literária}\label{cp:revisaoliteraria}
Foi realizada uma busca exploratória utilizando a ferramenta Google Scholar para encontrar artigos relevantes ao tema. A partir de \cite{kratochvila2024multi}, selecionado como referência inicial, foi possível gerar e refinar a string de busca.

\section{Busca Bibliográfica}\label{cp:refteory:figuras}
A busca nas bibliotecas arXiv e IEEE Xplore foi feita em setembro de 2025, abrangendo o período de 2021 a 2025, com as strings de busca mostradas a seguir:
   
\subsection{arXiv}
A biblioteca Python \texttt{arxiv}~\cite{pypi_arxiv} foi utilizada para realizar a busca no 
repositório arXiv.


String de busca no arXiv: 
\footnotesize\begin{verbatim}
('(all:"floor plan" OR all:floorplan OR all:floorplans 
    OR all:"architectural layout" OR all:"building layout") ' 
'AND (all:"3D reconstruction" OR all:"3D model" 
    OR all:"layout reconstruction" OR all:"vectorization" 
    OR all:"raster-to-vector" OR all:"plan-to-3D" 
    OR all:"scene generation" OR all:"3D scenes" OR all:Plan2Scene)'
'AND (all:"deep learning" OR all:"semantic segmentation" 
OR all:"multi-task" OR all:"multi-task learning" 
    OR all:"graph neural network" OR all:GNN 
    OR all:CNN OR all:YOLO OR all:DeepLab 
    OR all:"geometric optimization") '
'AND submittedDate:[20190101 TO 20251231] '
'ANDNOT (all:"point cloud" OR all:LiDAR OR all:LIDAR  
    OR all:SLAM OR all:mapping OR all:scan 
    OR all:BIM OR all:robot OR all:navigation 
    OR all:"scene understanding" OR all:"depth estimation")')
\end{verbatim}
\normalsize
\subsection{IEEE Xplore}
String de busca no IEEE Xplore:

O site do IEEE Xplore foi usado para realizar a busca com a string:
\footnotesize\begin{verbatim}
("3D reconstruction" OR "3D modeling" OR "3D generation") 
AND ("floor plan" OR "architectural plan") 
AND ("deep learning" OR "semantic segmentation")
\end{verbatim}
\normalsize

%Artigos do IEEE Xplore que não tive acesso foram excluidos
\section{Síntese dos Resultados da Busca}

O quadro a seguir apresenta o quantitativo de resultados obtidos e filtrados durante as etapas de busca e triagem.

\begin{quadro}[H]
\centering
\caption{Quantitativo das etapas de busca 
e triagem bibliográfica (2021\\-\\2025)}
\label{quadro:busca_invertido}
\begin{tabular}{|l|c|c|c|c|}
\hline
\textbf{Etapa} & \textbf{arXiv} & \textbf{IEEE Xplore}  & \textbf{Total Consolidado} \\
\hline
Total Recuperado & 56 & 14 & 70 \\
Após Título e/ou Abstract & 11 & 3 & 14 \\
Após Leitura Completa & 2 & 1 & 3 \\
\hline
Incluídos & 2 & 1 & 3 \\
\hline
\end{tabular}
\legend{Fonte: Autor}
\end{quadro}


Os artigos selecionados foram: \citeonline{kratochvila2024multi} e \citeonline{barreiro2023automatic} do ArXiv e \citeonline{lv2021residential} do IEEE Xplore. 

O artigo \citeonline{3dplanet2021} foi encontrado pelas referências de \citeonline{kratochvila2024multi} e \citeonline{barreiro2023automatic}.

\section{Análise dos Artigos}

\citeonline{kratochvila2024multi} desenvolveram dois algoritmos de segmentação, denominados CAB1 e CAB2, que utilizam um mecanismo de atenção \textit{(Attention Mechanism)} composto por CAM e SAM. Na etapa de reconstrução, cada categoria de pixel segmentada foi vetorizada para então gerar o modelo 3D. Foram utilizadas 560 imagens escolhidas do \textit{dataset} CubiCasa5K.
 A \autoref{fig:kratochvila2024} mostra a visão geral apresentada pelos autores.

\citeonline{lv2021residential} realizaram a detecção de áreas de interesse, textos e símbolos (móveis, pias) utilizando YOLOv4 \cite{bochkovskiy2020yolov4}.
O método calcula a escala do desenho, pixel por milímetro, enquanto outros, como o de \cite{3dplanet2021}, utilizaram um tamanho padrão.
Foi empregado um \textit{dataset} próprio com 7.000 imagens de residências chinesas.
A \autoref{fig:lv2021} mostra a visão geral apresentada pelos autores.

\citeonline{barreiro2023automatic} destacam a importância de focar apenas em paredes, portas e janelas, pois as anotações referentes ao tipo de cômodo são específicas de cada \textit{dataset} e dificultam a criação de uma técnica generalizada. 
A sequência de etapas proposta envolve a detecção de símbolos para identificar portas e janelas, seguida da segmentação de paredes utilizando arquitetura \textit{FPN} com o \textit{backbone} \textit{ResNet}, semelhante à de \cite{lv2021residential}.
A \autoref{fig:barreiro2023automatic} mostra a visão geral apresentada pelos autores.

\citeonline{3dplanet2021}, citados por \citeonline{kratochvila2024multi}, utilizaram o algoritmo de \cite{ijgi9020065} para detectar o centro das paredes. 
O método emprega a \textit{API} do \textit{Tensorflow} para a detecção de objetos: tipo de cômodo, portas e janelas. 
Foram 30 imagens escolhidas aleatoriamente, provenientes do \textit{dataset} de \citeonline{ijgi9020065}, que contém 110.000 imagens.
A \autoref{fig:3dplanet2021} mostra a visão geral apresentada pelos autores.

\begin{figure}[H]
\centering
\caption{Arquiteturas de Kratochvila et al. (2024) e Lv et al. (2021)}
\begin{subfigure}{0.47\textwidth}
  \centering
  \includegraphics[width=\linewidth]{imagens/kratochvila2024multi_figure4.png}
  \caption{Kratochvila et al. (2024)}
  \label{fig:kratochvila2024}
\end{subfigure}
\hfill
\begin{subfigure}{0.47\textwidth}
  \centering
  \includegraphics[width=\linewidth]{imagens/Lv_2021_figure2.png}
  \caption{Lv et al. (2021)}
  \label{fig:lv2021}
\end{subfigure}
\legend{Fonte: (a) \cite{kratochvila2024multi}, p. 5; (b) \cite{lv2021residential}, p. 16719.}
\label{fig:arquiteturas1}
\end{figure}

\begin{figure}[H]
\centering
\caption{Arquiteturas de Barreiro et al. (2023) e Park e Kim (2021)}
\begin{subfigure}{0.47\textwidth}
  \centering
  \includegraphics[width=\linewidth]{imagens/barreiro2023automatic_figure1.png}
  \caption{Da direita para a esquerda, planta baixa, máscara segmentada, resultados vetorizados e modelo 3D}
  \label{fig:barreiro2023automatic}
\end{subfigure}
\hfill
\begin{subfigure}{0.47\textwidth}
  \centering
  \includegraphics[width=\linewidth]{imagens/3dplanet2021_figure1.png}
  \caption{Park e Kim (2021)}
  \label{fig:3dplanet2021}
\end{subfigure}
\legend{Fonte: (a) \cite{barreiro2023automatic}, p. 3; (b) \cite{3dplanet2021}, p. 3.}
\label{fig:arquiteturas2}
\end{figure}

\section{Hardware dos Artigos}
A \autoref{tab:hardware-artigos} apresenta um resumo do hardware utilizado nos artigos analisados.
Essa informação é importante para avaliar a viabilidade de reprodução dos métodos descritos na literatura, considerando as limitações e capacidades do hardware disponível para este projeto.

\begin{table}[H]
\centering
\caption{Resumo do \textit{hardware} utilizado nos trabalhos analisados}
\label{tab:hardware-artigos}
\begin{tabular}{|l|p{6cm}|}
\hline
\textbf{Artigos} & \textbf{Hardware Utilizado} \\
\hline
\citeonline{kratochvila2024multi}  &
2× NVIDIA RTX A5000 GPU. \\
\hline
\citeonline{lv2021residential} &
Não informado pelos autores. \\
\hline
\citeonline{barreiro2023automatic}   &
Não informado pelos autores. \\
\hline
\citeonline{3dplanet2021}  &
Notebook com CPU i7-875H 2.20 GHz e GPU móvel
(entre GTX 1050 e RTX 2080 Max-Q). \\
\hline
\end{tabular}
\legend{Fonte: Autor}
\end{table}

% precisa apontar para o parágrafo da fonte? E como que escreve isso LaTeX

% Este capítulo define o protocolo de revisão utilizado para obtenção de trabalhos com propostas semelhantes as definidas nesse estudo, tal como a fundamentação teórica necessária para os conceitos apresentados no \autoref{cp:refteory}.

% \section{Protocolo de Revisão}\label{cp:revisao:protocolo}

% Foi usando a ferramenta Rayyan\cite{Rayyn} para fazer a filtragem dos artigos.

% \subsection{Critérios de Exclusão}
% \begin{itemize}
%     \item LiDAR - \textit{light detection and ranging}.
%     \item Slam  - \textit{Simultaneous localization and mapping}.
%     \item Point Clouds - Nuvem de pontos.    
% \end{itemize}

\section{Considerações Finais}
% Protocolo de obtenção dos artigos
% Frameworks com abordagens diferentes com foco na promoção do engajamento e motivação do usuário.
% Neste capítulo foram apresentados o protocolo de revisão e os trabalhos relacionados com os \nameref{cp:intro:obj} deste estudo. ....

Na busca bibliográfica, \cite{kratochvila2024multi} tornou-se o artigo principal de referência, pois cita e relaciona os demais trabalhos encontrados. A partir dele, foi possível identificar os \textit{datasets} e as técnicas predominantes aplicadas atualmente na reconstrução 3D a partir de plantas baixas. Observou-se também que os trabalhos analisados não disponibilizaram o código-fonte, o que dificulta a reprodutibilidade e a comparação direta entre abordagens.
