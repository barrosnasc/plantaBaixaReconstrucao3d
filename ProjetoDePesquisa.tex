
% ------------------------------------------------------------------------
% ------------------------------------------------------------------------
% abnTeX2: Modelo de Trabalho Academico (tese de doutorado, dissertacao de
% mestrado e trabalhos monograficos em geral) em conformidade com 
% ABNT NBR 14724:2011: Informacao e documentacao - Trabalhos academicos -
% Apresentacao
% MODELO DE PROJETO DE PESQUISA - UESPI
% ------------------------------------------------------------------------
% 


\documentclass[
	% -- opções da classe memoir --
	12pt,				% tamanho da fonte
	openright,			% capítulos começam em pág. ímpar (insere página vazia caso preciso)
	oneside,			% para impressão em verso e anverso. Oposto a oneside
	a4paper,			% tamanho do papel. 
	% -- opções da classe abntex2 --
	chapter=TITLE,		% títulos de capítulos convertidos em letras maiúsculas
	%section=TITLE,		% títulos de seções convertidos em letras maiúsculas
	%subsection=TITLE,	% títulos de subseções convertidos em letras maiúsculas
	%subsubsection=TITLE,% títulos de subsubseções convertidos em letras maiúsculas
	% -- opções do pacote babel --
	english,			% idioma adicional para hifenização
	%french,				% idioma adicional para hifenização
	%spanish,			% idioma adicional para hifenização
	brazil,				% o último idioma é o principal do documento
	]{abntex2}
% ---
% Pacotes básicos 
% ---
%\usepackage{lmodern}			% Usa a fonte Latin Modern	
\usepackage{mathptmx}
\renewcommand{\ABNTEXchapterfont}{\normalfont}
%\usepackage{bookman}		
\usepackage[T1]{fontenc}		% Selecao de codigos de fonte.
\usepackage[utf8]{inputenc}		% Codificacao do documento (conversão automática dos acentos)
\usepackage{lastpage}			% Usado pela Ficha catalográfica
\usepackage{indentfirst}		% Indenta o primeiro parágrafo de cada seção.
\usepackage{color}				% Controle das cores
\usepackage{graphicx}			% Inclusão de gráficos
\usepackage{microtype} 			% para melhorias de justificação
\usepackage{subcaption} %subcaptions em subfigures
\usepackage{tkz-graph} %diagrama

\usepackage{epigraph}

%Depois no documento escrevemos:

%\epigraph{texto}{referência}


%\usepackage{geometry}
%\geometry{a4paper, left=3cm, right=2cm, bottom=2cm, top=3cm}
% ---
\graphicspath{ {./figs/} }
% ---
% Pacotes adicionais, usados apenas no âmbito do Modelo Canônico do abnteX2
% ---
\usepackage{lipsum}				% para geração de dummy text
% ---
% Pacotes de citações
% ---
\usepackage[brazilian,hyperpageref]{backref}	 % Paginas com as citações na bibl
\usepackage[alf]{abntex2cite}	% Citações padrão ABNT
% Pacotes adicionais ++
\usepackage{listings}

% --- 
% CONFIGURAÇÕES DE PACOTES
% --- 

% ---
% Configurações do pacote backref
% Usado sem a opção hyperpageref de backref
\renewcommand{\backrefpagesname}{Citado na(s) página(s):~}
% Texto padrão antes do número das páginas
\renewcommand{\backref}{}
% Define os textos da citação
\renewcommand*{\backrefalt}[4]{
	\ifcase #1 %
		Nenhuma citação no texto.%
	\or
		Citado na página #2.%
	\else
		Citado #1 vezes nas páginas #2.%
	\fi}%
% ---


%%%%%comandos para revisão
\newcommand{\supervisor}[1]{\textcolor{red}{[#1]}}

\newcommand{\aluno}[1]{\textcolor{blue}{[#1]}}


%================================================================================
% Pacote para criacao da lista de siglas e abreviaturas
%================================================================================
\usepackage{nomencl}
\makeatletter
\setlength{\nomlabelwidth}{0.15\hsize} 
\renewcommand{\nomlabel}[1]{#1 \hfill}
\setlength{\nomitemsep}{-.05in} 
\makenomenclature
% Traduz o titulo da lista de abreviaturas
\renewcommand{\nomname}{Lista de Abreviaturas e Siglas}

% Comando para criacao de abreviaturas e siglas
\def\sigla{\@ifstar\@sigla\@@sigla}
% Apenas faz o registro na tabela de siglas
\def\@sigla#1#2{\nomenclature{#1}{#2}}          % \sigla*{}{} 
% Faz o registro na tabela de siglas e insere os dados no corpo do documento
\def\@@sigla#1#2{#2 (#1)\nomenclature{#1}{#2}}  % \sigla{}{}
\makeatother
%%%%%%%%%%%
%%%Lista de quadros


\usepackage{float}
\floatstyle{plaintop} % Coloca caption no topo
\newfloat{quadro}{htbp}{lop}
\floatname{quadro}{Quadro}
\newcommand{\listofquadros}{\listof{quadro}{Lista de Quadros}}
\def\quadroautorefname{Quadro}
%%%Lista de quadros




% ---
% Informações de dados para CAPA e FOLHA DE ROSTO
% ---
\titulo{Geração de Modelo 3D a partir de Planta Baixas utilizando Processamento de Imagem}
\autor{João Pedro Barros do Nascimento}
\local{Teresina}
\data{2025}
\orientador{Carlos Giovanni Nunes de Carvalho}
\instituicao{Universidade Estadual do Piauí}
\tipotrabalho{Projeto de Pesquisa (Graduação)}

% O preambulo deve conter o tipo do trabalho, o objetivo, o nome da instituição e a área de concentração 
\preambulo{Pré-projeto de Trabalho de Conclusão de Curso apresentado na Universidade Estadual do Piauí – UESPI como parte dos requisitos para conclusão do Curso de Bacharelado em Ciência da Computação.}



% Configurações de aparência do PDF final
%\definecolor{blue}{RGB}{41,5,195}
% informações do PDF
\makeatletter
\hypersetup{
     	%pagebackref=true,
		pdftitle={\@title}, 
		pdfauthor={\@author},
    	pdfsubject={\imprimirpreambulo},
	    pdfcreator={LaTeX with abnTeX2},
		pdfkeywords={abnt}{latex}{abntex}{abntex2}{trabalho acadêmico}, 
		colorlinks=true,       		% false: boxed links; true:              colored links
    	linkcolor=black,          	% color of internal links
    	citecolor=black,        		% color of links to bibliography
    	filecolor=magenta,      		% color of file links
		urlcolor=black,
		bookmarksdepth=4}

\makeatother
% --- 
% Espaçamentos entre linhas e parágrafos 
% --- 
% O tamanho do parágrafo é dado por:
\setlength{\parindent}{1.3cm}
% Controle do espaçamento entre um parágrafo e outro:
\setlength{\parskip}{0.2cm}  % tente também \onelineskip
% ---
% compila o indice
% ---
\makeindex
% ---

% --------------------------------------------------------------------------------------------------------
% % Início do documento
% --------------------------------------------------------------------------------------------------------
\begin{document}

% Retira espaço extra obsoleto entre as frases.
\frenchspacing 

%%capa
% Capa
% A capa no projeto de pesquisa é opcional
%---------------------------------------------------------------------------------------------------------

    \begin{center}
    {\ABNTEXchapterfont\large UNIVERSIDADE ESTADUAL DO PIAUÍ}\\
    {\ABNTEXchapterfont\large CIÊNCIA DA COMPUTAÇÃO}
    
    \vspace*{\fill}\vspace*{\fill}
    {\ABNTEXchapterfont\large \imprimirautor}  
    \vspace*{\fill}
   
    \vspace*{\fill}\vspace*{\fill}
    \begin{center}
    \ABNTEXchapterfont \bfseries \Large \imprimirtitulo
    \end{center}
    \vspace*{\fill}\vspace*{\fill}
   
  
   \end{center}  

      
    \begin{center}
    \vspace*{0.5cm}
    {\ABNTEXchapterfont\large TERESINA}
    \par
    {\ABNTEXchapterfont\large \imprimirdata}
    \vspace*{1cm}
    \end{center}
%--------------------------------------------------x------------------------------------------------------

%%



% Folha de Rosto
% --------------------------------------------------------------------------------------------------------
\begin{folhaderosto}

  \begin{center}
    {\ABNTEXchapterfont\large \imprimirautor}

    \vspace*{\fill}\vspace*{\fill}
    \begin{center}
      \ABNTEXchapterfont\bfseries\Large \imprimirtitulo
    \end{center}
    \vspace*{\fill}
    
    \hspace{.45\textwidth}
    \begin{minipage}{.5\textwidth}
        \imprimirpreambulo
    \end{minipage}%
    
    \vspace{1.5cm}
    Orientador: \imprimirorientador \\
    
    \vspace*{\fill}
     
   \end{center}  

     \begin{center}
    \vspace*{0.5cm}
    {\ABNTEXchapterfont\large TERESINA}
    \par
    {\ABNTEXchapterfont\large \imprimirdata}
    \vspace*{1cm}
    \end{center}
  
\end{folhaderosto}
%--------------------------------------------------x----------------------------------------

% Resumo
%---------------------------------------------------------------------------------------------------------
\setlength{\absparsep}{18pt} % ajusta o espaçamento dos parágrafos do resumo
\begin{resumo}
\ABNTEXchapterfont
Plantas baixas são difíceis de interpretar e oferecem pouca profundidade visual, enquanto sua transformação em um modelo 3D permite compreender melhor o conteúdo representado. Este trabalho propõe reconstruir um modelo 3D a partir de uma planta baixa, seguindo técnicas descritas na literatura. A metodologia envolve a coleta de plantas baixas da cidade de Teresina, o treinamento de um modelo capaz de extrair as características arquitetônicas presentes na imagem e a exibição do modelo 3D para comparação com o dado original. Espera-se, ao final, obter um sistema que receba uma planta baixa e apresente seu modelo 3D reconstruído.

\textbf{Palavras-chaves}: Processamento de Imagem; Modelo 3D; Reconstrução 3D; Planta baixa.
\end{resumo}
%--------------------------------------------------x------------------------------------------------------


% Abstract
%---------------------------------------------------------------------------------------------------------
\begin{resumo}[Abstract]
\ABNTEXchapterfont
Floor plans are often difficult to interpret and provide limited visual depth, while their conversion into a 3D model allows for a clearer understanding of the represented architectural content. This work proposes reconstructing a 3D model from a floor plan by following techniques described in the literature. The methodology involves collecting floor plans from the city of Teresina, training a model capable of extracting their architectural features, and displaying the resulting 3D model for comparison with the original input. The expected outcome is a functional system that receives a floor plan and presents its reconstructed 3D model.
% \begin{otherlanguage*}{english}
%   This is the english abstract.
%   \vspace{\onelineskip}
%   \noindent 

\textbf{Keywords}: Image Processing; 3D Model; 3D Reconstruction; Floor Plan.
\end{resumo}
%--------------------------------------------------x------------------------------------------------------


% inserir lista de ilustrações
%---------------------------------------------------------------------------------------------------------
\pdfbookmark[0]{\listfigurename}{lof}
\listoffigures*
\cleardoublepage
%--------------------------------------------------x------------------------------------------------------


% inserir lista de tabelas
%---------------------------------------------------------------------------------------------------------
% \pdfbookmark[0]{\listtablename}{lot}
% \listoftables*
% \cleardoublepage
%--------------------------------------------------x------------------------------------------------------


% inserir lista de abreviaturas e siglas
%-------------------------------------------------------------------------------------------
% para inserir as  abreviaturas e siglas use no texto o comando \sigla{SIGLA}{DEFINICAO DA SIGLA}a primeira vez que for definir a sigla 
%---------------------------------------------------------------------------------------------------------
\printnomenclature
\cleardoublepage
%--------------------------------------------------x------------------------------------------------------

%%%Lista de codigos
% Inclui a lista de códigos
% \incluilistadecodigos

%%Lista de quadros
\pdfbookmark[0]{\listtablename}{lot}
\listofquadros
\cleardoublepage

%--------------------------------------------------x------------------------------------------------------
\pdfbookmark[0]{\contentsname}{toc}
\tableofcontents*
\cleardoublepage
\textual
%--------------------------------------------------x------------------------------------------------------
                                          % INTRODUÇÃO %
%---------------------------------------------------------------------------------------------------------
\chapter{Introdução}\label{cp:introducao}
\ABNTEXchapterfont

% Contextualização problema em questão
% O que encontrei na bibliografia

% % Motivação do avanço tecnologico que motivou a utilização dessa técnica ao ínves das anteriores

% 2021Alibaba: Capítulo 2,¶1 e ¶2 :
% [...] Traditional methods [8, 9, 10] focus on directly processing low-level features. These systems produce a large number of hand-designed features and models. [...] Those systems mentioned above bring the problem of insufficient generalization ability. Thresholds and features are adjusted frequently by handcrafted operations instead of automatic methods. 
% With the development of deep learning techniques, the method of obtaining room structure has made significant process in generalization. Convolutional Neural Network (CNN) can create and extract advanced features to enhance the recognition performance of room elements [22, 29, 33]. Liu et al. [22] illustrate junctions of floor plans, for example, corners of walls, could be recognized by CNN [...]. However, the approach has limitations, for instance, it is not able to detect inclined walls. [...]. 

% - Justificativa digitalização dos documentos e utilização de
% dados quantitativos(monetário) na justificativo ( a lá redação enem )
% Dados quantitativos 
% MultiFloor: "Urban areas experience a steady growth, and over two-thirds of the population worldwide will be considered urbanized by the year 2050[1]" -> https://population.un.org/wup/assets/WUP2018-Report.pdf % Pegar a versão atualizada desse texto

% Dados qualitativos 
% MultiFloor: "may benefit multiple fields and subdomains, such as creating virtual twins of buildings [2], optimal evacuation path planning [7], and firefighting " 
% 2021Alibaba: "after the above rasterization process, designers cannot modify the structure of the room and redesign flexibly. Therefore, accurately recovering vectorized information from pixel images becomes an urgent problem to be solved."
% Precisa de uma referência que cita projetos de renovação

% Dados que justificam o uso da planta 3D ao invés da 2D
% ex: diminuição de custo, segurança pública

% Escrita
% Premissa: Há uma necessidade de utilizar gerar modelos 3D a partir de plantas 2D

O desenho arquitetônico contém vários tipos de objetos: paredes, portas, janelas, quartos, nomes de cômodos, móveis e suas dimensões, entre outros. 
Essas características fornecem ao leitor os meios para reproduzir, no mundo real, o que está representado no papel, seja em construções ou em simulações. 
A recuperação do modelo arquitetônico a partir de uma planta baixa é uma necessidade de arquitetos e designers que desejam modificar ou alterar um projeto \cite{lv2021residential}. Há também a necessidade de criação de modelos 3D para o planejamento de rotas de fuga e combate a incêndios \cite{kratochvila2024multi}.
%, atualmente os projetos \iffalse achismo sem referencia \fi são feitos com o auxilio do computador conhecido como \sigla{CAD}{Computer Aided Design}.

A transferência do desenho arquitetônico para a construção é estática e não pode ser alterada, o que gera um problema caso seja necessário realizar uma reforma. Torna-se, portanto, necessária a utilização de um ambiente que permita alterar a organização dos objetos na planta, algo viável no ambiente virtual \cite{lv2021residential}.
% transição de somente a extração de características do modelo 2D para um modelo CAD funcional para o modelo 3D, o que se ganha ao fazer esse tipo de transfomação?
Na ausência de um modelo virtual do desenho arquitetônico, é possível realizar sua reconstrução, seja manualmente, seja por meio do processamento de imagem. 
Neste projeto, será utilizada a técnica de processamento de imagem, que extrai objetos e características do documento digitalizado, transfere-os para um ambiente virtual, processa-os para gerar uma estrutura de dados manipulável e, por fim, transforma-os em um modelo 3D \cite{yang2022automated}.

As primeiras técnicas de processamento de imagem para extração de características de plantas baixas utilizavam algoritmos heurísticos para obter informações das imagens. Com o avanço da ciência, observou-se que técnicas de aprendizado profundo apresentam desempenho superior \cite{lv2021residential}. Essas técnicas exigem \textit{datasets} para o treinamento dos modelos de predição, como o CubiCasa5K \cite{cubiCasa5K}, que contém 5 mil imagens de plantas baixas de residências da Finlândia.

\section{Objetivos}\label{cp:intro:obj}
Considerando a contextualização apresentada, nesta seção são apresentados os objetivos gerais e específicos deste projeto de pesquisa.

\subsection{Objetivo Geral}\label{cp:intro:objgeral}
Desenvolver um programa que utilize técnicas de processamento de imagem e detecção de símbolos aplicadas a plantas baixas, com base nas abordagens apresentadas nos artigos de referência, para gerar uma estrutura de dados que possibilite a criação de um modelo 3D. % das quais utilizando técnicas de baixo custo de hardware\cite{yang2022automated}.

\subsection{Objetivos Específicos}\label{cp:intro:objespec}
\begin{itemize}
  \item Revisar a bibliografia para compreender melhor as técnicas apresentadas nos artigos de referência e, com base nessa revisão, recriar a arquitetura utilizada.
  \item Escolher a arquitetura a ser utilizada, priorizando aquela com melhor compatibilidade com os requisitos de hardware disponíveis.
  \item Treinar o modelo de acordo com os \textit{datasets} da literatura.
  \item Coletar plantas baixas, preferencialmente de Teresina, para serem processadas e utilizadas na validação do modelo. % Digitalizar as plantas baixas disponíveis de Teresina.
  \item Fazer a inferência do modelo com as plantas baixas selecionadas. % Executar o modelo e exportar para um formato de arquivo 3D: $.obj$ ou $.gltf$
  \item Reconstruir e exibir o modelo 3D, para visualizar e comparar com a planta original.%  Utilizando o framework ThreeJS \iffalse baseado em qual artigo que a escolhar do ThreeJS é valida? os artigos utilizam alguma forma de exibição mas não disseram qual e como especificamente \fi exibindo o desenho 3D, no navegador com o controle de órbita e $panning$.
\end{itemize}


\section{Organização do Trabalho}\label{cp:intro:organization}
Esta monografia está estruturada em 4 capítulos. O \autoref{cp:introducao} tem como objetivo mostrar ao leitor um panorama geral do trabalho, incluindo a contextualização ao qual a pesquisa está inserida, a justificativa e os objetivos geral e específicos. O \autoref{cp:refteory} apresenta o referencial teórico necessário para fundamentar os principais conceitos diretamente relacionados com a pesquisa a ser desenvolvida. No \autoref{cp:revisaoliteraria} são apresentados o protocolo de revisão literária, tal como os trabalhos relacionados com a proposta dessa pesquisa. No \autoref{cp:metodologia} são apresentados a classificação da pesquisa, o método a ser aplicado, o cronograma de desenvolvimento da monografia, o fluxograma das etapas de desenvolvimento assim como os resultados que se espera alcançar através desta pesquisa. 

O trabalho é finalizado com a apresentação das referências bibliográficas que estruturaram a apresentação dos conceitos que constituem este estudo.


%--------------------------------------------------x------------------------------------------------------
\chapter{Referencial Teórico}\label{cp:refteory}
\ABNTEXchapterfont
% Justificativa das tecnologias, comparação 
%Este capítulo é um componente importante para esse estudo, pois fornece uma base  explorando as principais teorias, conceitos e definições existentes sobre as áreas de conhecimento que motivaram a elaboração deste projeto de pesquisa e que guiarão a execução do percurso metodológico.

%Escreva sobre o referencial teórico do seu trabalho

% Qual é o tipo de problema
% Utilizando o artigo do 3DPlanNet

Como já citado, o desenho arquitetônico contém vários tipos de objetos: paredes, portas, janelas, quartos, nomes de cômodos, móveis e suas dimensões, entre outros. Cada autor opta por dar ênfase a determinados desses aspectos. Seguindo a abordagem de \citeonline{kratochvila2024multi} e \citeonline{lv2021residential}, a implementação é dividida em duas etapas: reconhecimento e reconstrução.
Na etapa de reconhecimento ocorre o processamento de imagem, gerando um modelo intermediário, e na reconstrução são realizados o pós-processamento e a criação do modelo 3D, com pelo menos os objetos de parede, porta e janela.

No reconhecimento há a detecção de paredes, portas e janelas, sendo criada uma rede de vértices e arestas das paredes \cite{3dplanet2021}. Dependendo do autor, pode haver detecção dos outros tipos de objetos presentes na planta.
Na reconstrução é feito o pós-processamento para reduzir erros e, em seguida, é criado o modelo 3D usando tamanhos padrão para os objetos, como altura e largura da parede. Caso a técnica permita detectar outros elementos na etapa anterior, eles são adicionados ao modelo 3D.



\section{Trabalhos Relacionados}

\citeonline{kratochvila2024multi} desenvolveram dois algoritmos de segmentação, denominados CAB1 e CAB2, que utilizam um mecanismo de atenção \textit{(Attention Mechanism)} composto por CAM e SAM. Na etapa de reconstrução, cada categoria de pixel segmentada foi vetorizada, para, então gerar o modelo 3D. Foram utilizadas 560 imagens escolhidas do \textit{dataset} CubiCasa5K.
 A \autoref{fig:kratochvila2024} mostra a visão geral apresentada pelos autores.

\citeonline{lv2021residential} realizaram a detecção de áreas de interesse, textos e símbolos (móveis, pias) utilizando YOLOv4 \cite{bochkovskiy2020yolov4}.
O método calcula a escala do desenho, pixel por milímetro, enquanto outros, como o de \cite{3dplanet2021}, utilizaram um tamanho padrão.
Foi empregado um \textit{dataset} próprio com 7.000 imagens de residências chinesas.
A \autoref{fig:lv2021} mostra a visão geral apresentada pelos autores.

\citeonline{barreiro2023automatic} destacam a importância de focar apenas em paredes, portas e janelas, pois as anotações referentes ao tipo de cômodo são específicas de cada \textit{dataset} e dificultam a criação de uma técnica generalizada. 
A sequência de etapas proposta envolve a detecção de símbolos para identificar portas e janelas, seguida da segmentação de paredes utilizando arquitetura \textit{FPN} com o \textit{backbone} \textit{ResNet}, semelhante à de \cite{lv2021residential}. 
A \autoref{fig:barreiro2023automatic} mostra a visão geral apresentada pelos autores.

\citeonline{3dplanet2021}, citados por \citeonline{kratochvila2024multi}, utilizaram o algoritmo de \cite{ijgi9020065} para detectar o centro das paredes. 
O método emprega a \textit{API} do \textit{Tensorflow} para a detecção de objetos: tipo de cômodo, portas e janelas. 
Foram 30 imagens escolhidas aleatoriamente, provenientes do \textit{dataset} de \citeonline{ijgi9020065}, que contém 110.000 imagens.
A \autoref{fig:3dplanet2021} mostra a visão geral apresentada pelos autores.

\begin{figure}[H]
\centering
\caption{Arquiteturas de Kratochvila et al. (2024) e Lv et al. (2021)}
\begin{subfigure}{0.47\textwidth}
  \centering
  \includegraphics[width=\linewidth]{imagens/kratochvila2024multi_figure4.png}
  \caption{Kratochvila et al. (2024)}
  \label{fig:kratochvila2024}
\end{subfigure}
\hfill
\begin{subfigure}{0.47\textwidth}
  \centering
  \includegraphics[width=\linewidth]{imagens/Lv_2021_figure2.png}
  \caption{Lv et al. (2021)}
  \label{fig:lv2021}
\end{subfigure}
\legend{Fonte: (a) \cite{kratochvila2024multi}, p. 5; (b) \cite{lv2021residential}, p. 16719.}
\label{fig:arquiteturas1}
\end{figure}

\begin{figure}[H]
\centering
\caption{Arquiteturas de Barreiro et al. (2023) e Park e Kim (2021)}
\begin{subfigure}{0.47\textwidth}
  \centering
  \includegraphics[width=\linewidth]{imagens/barreiro2023automatic_figure1.png}
  \caption{Da direita para a esquerda, planta baixa, máscara segmentada, resultados vetorizados e modelo 3D}
  \label{fig:barreiro2023automatic}
\end{subfigure}
\hfill
\begin{subfigure}{0.47\textwidth}
  \centering
  \includegraphics[width=\linewidth]{imagens/3dplanet2021_figure1.png}
  \caption{Lv et al. (2021)}
  \label{fig:3dplanet2021}
\end{subfigure}
\legend{Fonte: (a) \cite{barreiro2023automatic}, p. 3; (b) \cite{3dplanet2021}, p. 3.}
\label{fig:arquiteturas2}
\end{figure}


% colocar a figura do artigo

% O processamento da imagem gera alguns problemas sendo que antes foram solucionados por algoritmos heurísticos, mas atualmente as soluções usam técnicas de aprendizagem profunda \iffalse referênciar? \fi e essa mudança mostrou uma melhora na acurácia na detecção dos objetos \cite{3dplanet2021}.

% Um dos problemas que isso gera é a necessidade da criação de datasets devidamente anotados para treinar os modelos de aprendizagem profundas.

% Utilizar o resumo/conclusão de cada artigo, e discutir sobre os métodos e os achados e comentar sobre os artigos mostrando que ainda há algo a ser feito 

% Como as pessoas estão resolvendo os problemas
% sugestão de tabelinha de como comparar as diferentes soluções ajuda consideravelmente no entendimento do trabalho

\section{Considerações Finais}
Os trabalhos analisados utilizam técnicas de processamento de imagem e reconstrução 3D, mas a indisponibilidade de código-fonte dificulta sua reprodução.

 % Desenvolvimento %
%---------------------------------------------------------------------------------------------------------
\chapter{Revisão Literária}\label{cp:revisaoliteraria}
Foi feito uma busca exploratoria utilizando a ferramenta Google Scholar para encontrar um artigo que fosse relevante ao tema e foi encontrado \cite{kratochvila2024multi}, com o conteúdo desse foi gerado e refinado a string de busca.

\section{Busca Bibliográfica}\label{cp:refteory:figuras}
A busca nas bibliotecas: arXiv e IEEExplore foi feita em Setembro de 2025, sobre os períodos de 2021 e 2025 e com strings de busca mostrado a seguir:
   
\subsection{arXiv}
A bibloteca python https://pypi.org/project/arxiv/ , foi utilizada para fazer a busca no arXiv com a string de busca.

String de Busca arXiv: 
\begin{verbatim}
('(all:"floor plan" OR all:floorplan OR all:floorplans 
    OR all:"architectural layout" OR all:"building layout") ' 
'AND (all:"3D reconstruction" OR all:"3D model" 
    OR all:"layout reconstruction" OR all:"vectorization" 
    OR all:"raster-to-vector" OR all:"plan-to-3D" 
    OR all:"scene generation" OR all:"3D scenes" OR all:Plan2Scene)'
'AND (all:"deep learning" OR all:"semantic segmentation" 
OR all:"multi-task" OR all:"multi-task learning" 
    OR all:"graph neural network" OR all:GNN 
    OR all:CNN OR all:YOLO OR all:DeepLab 
    OR all:"geometric optimization") '
'AND submittedDate:[20190101 TO 20251231] '
'ANDNOT (all:"point cloud" OR all:LiDAR OR all:LIDAR  
    OR all:SLAM OR all:mapping OR all:scan 
    OR all:BIM OR all:robot OR all:navigation 
    OR all:"scene understanding" OR all:"depth estimation")')
\end{verbatim}
\subsection{IEEExplore}
String de Busca IEEExplore:

O site do IEEE Explore foi usado para fazer a a busca com a string de busca.
\begin{verbatim}
("3D reconstruction" OR "3D modeling" OR "3D generation") 
AND ("floor plan" OR "architectural plan") 
AND ("deep learning" OR "semantic segmentation")
\end{verbatim}

\begin{quadro}[H]
\centering
\caption{Quantitativo das etapas de busca 
e triagem bibliográfica (2021\\-\\2025)}
\label{quadro:busca_invertido}
\begin{tabular}{|l|c|c|c|c|}
\hline
\textbf{Etapa} & \textbf{arXiv} & \textbf{IEEE Xplore}  & \textbf{Total Consolidado} \\
\hline
Total Recuperado & 56 & 14 & 70 \\
Após Título e/ou Abstract & 11 & 3 & 14 \\
Após Leitura Completa & 2 & 1 & 3 \\
\hline
Incluídos & 2 & 1 & 3 \\
\hline
\end{tabular}
\legend{Fonte: Autor}
\end{quadro}

%Artigos do IEEExplore que não tive acesso foram excluidos

Os artigos selecionados foram: \citeonline{kratochvila2024multi} e \citeonline{barreiro2023automatic} do ArXiv e \citeonline{lv2021residential} do IEEEexplore. 

Os outros artigos citados nesse projeto de pesquisa foram encontrados através das suas referências.

% precisa apontar para o parágrafo da fonte? E como que escreve isso LaTeX

% Este capítulo define o protocolo de revisão utilizado para obtenção de trabalhos com propostas semelhantes as definidas nesse estudo, tal como a fundamentação teórica necessária para os conceitos apresentados no \autoref{cp:refteory}.

% \section{Protocolo de Revisão}\label{cp:revisao:protocolo}

% Foi usando a ferramenta Rayyan\cite{Rayyn} para fazer a filtragem dos artigos.

% \subsection{Critérios de Exclusão}
% \begin{itemize}
%     \item LiDAR - \textit{light detection and ranging}.
%     \item Slam  - \textit{Simultaneous localization and mapping}.
%     \item Point Clouds - Nuvem de pontos.    
% \end{itemize}

\section{Considerações Finais}
% Protocolo de obtenção dos artigos
% Frameworks com abordagens diferentes com foco na promoção do engajamento e motivação do usuário.
% Neste capítulo foram apresentados o protocolo de revisão e os trabalhos relacionados com os \nameref{cp:intro:obj} deste estudo. ....

Na busca bibliográfica  \cite{kratochvila2024multi} se tornou o artigo principal de referência, ele referência os demais artigos que encontrei na busca bibliográfica, com ele pode identificar os datasets e quais são as técnicas utilizados atualmente.
%--------------------------------------------------x------------------------------------------------------

 % Conclusões e Trabalhos Futuros %
%---------------------------------------------------------------------------------------------------------
\chapter{Proposta e Metodologia}\label{cp:metodologia}
Para a formulação dos procedimentos de pesquisa, retoma-se o objetivo proposto neste estudo:
Irei usar a plaforma do Google Collab para o treinamento e inferência
Irei usar o dataset CubiCasa5k
Irei usar o Pytorch para a criação do modelo
Irei usar o 
\begin{quotation}
\textit{ --  }
\end{quotation}

Este capítulo visa relatar o processo metodológico utilizado para desenvolvimento do estudo (\autoref{cp:intro:metodos}), o cronograma de tarefas (\autoref{cp:proposta:cronograma}) e os resultados que se espera obter ao fim do prazo de execução da monografia (\autoref{resultados}).

\section{Método e Procedimentos de Pesquisa}\label{cp:intro:metodos}
 ....

Com base nessas informações, do ponto de vista dos objetivos, esse estudo pode ser considerado uma pesquisa .... As etapas de desenvolvimento do projeto de pesquisa ...:

\begin{enumerate} 
\item ...

\end{enumerate}

\section{Cronograma de Pesquisa}\label{cp:proposta:cronograma}
O \autoref{cronograma} apresenta o cronograma mensal de atividades necessárias para desenvolvimento da pesquisa proposta ...

Criação de figura 

\begin{quadro}[!htb]
    \centering
    \caption{Cronograma de desenvolvimento das atividades de pesquisa.}\label{cronograma}
    \begin{tabular}{|l|c|c|c|c|}
    \hline
    & \multicolumn{4}{|c|}{2025} \\ \cline{2-5}
    \textbf{Atividades}&  Janeiro&  Fevereiro&  Março& Abril\\ \hline  
    1. Rev. Biblio.&  X&  X&  X& X \\ \hline  
    2. Map. teo. BD&  X&  X&  & \\ \hline  
    3. Exp. Elementos Design&  X&  X&  & \\ \hline  
    4. Def. Elementos &  X&  X&  & \\ \hline  
    5. Prop. Métricas&  X&  X&  & \\ \hline  
    6. Desen. do Framework&  &  X&  X& \\ \hline  
    7. Util. Framework&  X&  &  X& X\\ \hline  
    8. Escrita&  &  X&  X& X\\ \hline  
    9. Defesa&  &  &  & X\\ \hline 
    \end{tabular}

    \vspace{1cm}
    
\end{quadro}

\section{Resultados Esperados}\label{resultados}
 


%--------------------------------------------------x------------------------------------------------------

%--------------------------------------------------x-----------------------------------------------------

%--------------------------------------------------x-----------------------------------------------------

%--------------------------------------------------x-----------------------------------------------------
% ----------------------------------------------------------
% ELEMENTOS PÓS-TEXTUAIS
% ----------------------------------------------------------
%{Referencias Bibliograficas}
%---------------------------------------------------------------------------------------------------------
\bibliography{biblio.bib}
\printindex

% ---------------------------------------------------------------------
    % GLOSSÁRIO
    % --------------------------------------------------------------------- 
    % Arquivo que contém as definições que vão aparecer no glossário
    %\input{tex/glossario}
    % Comando para incluir todas as definições do arquivo glossario.tex
    %\glsaddall
    % Impressão do glossário
    %\printglossaries

    % ----------------------------------------------------------
    % Apêndices
    % ----------------------------------------------------------
    
    % ---
    % Inicia os apêndices
    % ---
   % \apendices{
    %    \chapter{References Set}
    %    \label{apendice:setinicial}
    %    \input{trabRelacionados}

   % }
    % ---
    

    % ----------------------------------------------------------
    % Anexos
    % ----------------------------------------------------------
    
    % ---
    % Inicia os anexos
    % ---
    %\anexos{
    
    
    %\begin{anexosenv}
    
        
    %\end{anexosenv}
    %}
    % ---


\end{document}
